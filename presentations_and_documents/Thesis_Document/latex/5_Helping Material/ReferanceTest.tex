
\documentclass[runningheads,a4paper]{llncs}
\usepackage{amssymb}
\setcounter{tocdepth}{3}
\usepackage{url}


\begin{document}

\mainmatter  % start of an individual contribution

\section{The References Section}\label{references}

\begin{thebibliography}{10}

\bibitem{url1} National Center for Information, \url{http://uaps.ucf.edu/enrollment/methods_detailed.html}
\bibitem{jour1} Serge Herzog: Estimating Student Retention and Degree-Completion Time: Decision Trees and Neural Networks Vis-�-Vis Regression. New Directions for Institutional Research, vol 2006, Issue 131, pg. 17-33 (2006).
\bibitem{jour2} Renee Neely: Discriminant Analysis for Prediction of College Graduation. Educational and Psychological Measurement, vol. 37, (1977)
\bibitem{jour3} Cliff Sessoms: Statistical Analysis Research Papers. Statistical Analysis SPEA Vol-506, (2010)
\bibitem{url2} J. Luan: Data Mining Applications in Higher Education, \url{http://www.insol.lt/media/collateral/modeling/education.pdf}
\bibitem{url3} 2012, Enrollment \& Degree Statistics: MIT Office of the Registrar, \url{http://web.mit.edu/registrar/stats/index.html}
\bibitem{url4} 2012, Stanford University: Common Data Set 2011-2012, \url{ http://ucomm.stanford.edu/cds/2011.html}
\bibitem{proceeding1} Bruce D. Beck: Enrollment Forecasting and Modeling. UW-Madison November 12, (2009)
\bibitem{jour4} A. Nandeshwar, S. Chaudhari: Enrollment Prediction Models Using Data Mining. Retrieved January, vol. 10, (2009)
\bibitem{proceeding2} J. Luan and A. M. Serban: Data mining and its application in higher education. In Knowledge Management: Building a Competitive Advantage in Higher Education: New Directions for Institutional Research. Jossey-Bass, (2002)
\bibitem{jour5} J. M. B. Gonzlez and S. L. DesJardins: Articial neural networks: A new approachto predicting application behavior. Research in Higher Education, 43(2):235-258, (2002)
\bibitem{jour6} L. Chang: Applying data mining to predict college admissions yield: A case study. New Directions for Institutional Research, 2006(131), (2006)
\bibitem{jour7} C.M. Antons and E.N. Maltz: Expanding the role of institutional research at small private universities: A case study in enrollment management using data mining. New Directions for Institutional Research, 2006(131):69, (2006)
\bibitem{proceeding3} M. J. Druzdzel and C. Glymour: Application of the TETRAD II program to the study of student retention in u.s. colleges. In Working notes of the AAAI-94 Workshop on Knowledge Discovery in Databases (KDD-94), pages 419-430, Seattle, WA, (1994)
\bibitem{proceeding4} A.P. Sanjeev and J.M. Zytkow: Discovering enrolment knowledge in university databases. In First International Conference on Knowledge Discovery and Data Mining, pages 246-251, Montreal, Que., Canada, (1995)
\bibitem{proceeding5} S. Massa and P.P. Puliato: An application of data mining to the problem of the university students' dropout using markov chains. In Principles of Data Mining and Knowledge Discovery. Third European Conference, PKDD'99, pages 51-60, Prague,Czech Republic, (1999)
\bibitem{proceeding6} Chong Ho Yu, Samuel DiGangi, Angel Jannasch-Pennell, Wenjuo Lo, and Charles Kaprolet: A data-mining approach to differentiate predictors of retention between online and traditional students, (2007)
\bibitem{proceeding7} A. Salazar, J. Gosalbez, I. Bosch, R. Miralles, and L. Vergara: A case study of knowledge discovery on academic achievement, student desertion and student retention. Information Technology: Research and Education, 2004. ITRE 2004. 2nd International Conference on, pages 150-154, (2004)
\bibitem{proceeding8} D. L. Stewart and B. H. Levin: A model to marry recruitment and retention: A case study of prototype development in the new administration of justice program at blue ridge community college, (2001)
\bibitem{url5} J. Gonzalez: Graduation Rates Can Be Predicted More Precisely by Examining Student Characteristics, Report Says, 29 Nov. 2011, \url{http://chronicle.com/article/article-content/129914/}
\bibitem{url6} Mark J. Perry: 142 Women Enrolled in Grad School Per 100 Men, and Women Outnumber Men in 7 Out Of 11 Fields, 14 Sept. 2010, \url{http://mjperry.blogspot.com/2010/09/there-are-142-women-enrolled-in-grad.html}
\bibitem{book1} J. Han and M. Kamber: Data Mining: Concepts and Techniques. 2nd edition, Morgan Kaufmann Publishers (2005)

\end{thebibliography}



\end{document} 