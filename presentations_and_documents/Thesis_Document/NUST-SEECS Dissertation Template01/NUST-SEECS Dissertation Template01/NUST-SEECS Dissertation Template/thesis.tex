\documentclass[12pt,a4paper,oneside]{book} 
%scrbook book report

\usepackage{multirow}
\usepackage{graphicx}
\usepackage[table]{xcolor}
\usepackage{float}
%\usepackage{color}
%\usepackage{subfigure}
\usepackage{seecs}
%\usepackage{color}
%\usepackage{colortbl}
%\usepackage{soul}
%\usepackage{listings}
%\lstloadlanguages{Java,XML}
%\lstset{frame=lines}
%\usepackage{astron}
%\usepackage{xspace}
%\usepackage[leqno]{amsmath}
%\usepackage{hyperref}
%\usepackage{sfmath}
%\usepackage{setspace}
\usepackage{cite}
\definecolor{lightgray}{gray}{0.9}


%\usepackage{amsmath,amssymb,amsthm,paralist}
%% Include other packages you wish to use except setspace.
%% That package is loaded automatically.
%% IMPORTANT: Load only those packages you know you will use.
%% Some packages can cause conflicts resulting in improper formatting.
%\usepackage{graphicx}
%\usepackage{subfig}
%\usepackage{latexsym}
%\usepackage{url}
%%\usepackage{soul}  % package for strikeout \st{}
%\usepackage{algorithm}
%\usepackage{algorithmic}
%\usepackage{cite}


\title{Title of Your Dissertation}
%\subtitle{An optional sub-title, usually not used at NUST}

\author{Your Name Goes Here}
\regno{2013-NUST-MS-EE-xx}
\degree{\MSEE} 
% Argument pptions for \degree{_____}:
% \BSIT for Bachelor of Science in Information Technology (BS IT)
% \BSCS for Bachelor of Science in Computer Science (BS CS)
% \BICSE for Bachelor of Engineering in Information and Communication Systems (BE ICS)
% \BEE for Bachelor of Engineering in Electronics (BE Electronics)
% \MSIT for Masters of Science in Information Technology (MS IT)
% \MSCSE for Masters in Communication Systems Engineering (MS CSE)
% \MSCCS for Masters in Computer and Communication Security (MS CCS)
% \MSEE for Masters of Science in Electrical Engineering (MS EE)

\adviser{Dr. FirstName LastName}
\adviserAffiliation{Department of Electrical Engineering}

\date{April 2015}

%\setcounter{tocdepth}{2}
 %\setstretch{1.1}
 %\linespread{1.1}

\begin{document}
\maketitle

\evaluationcommitteeapproval{Dr. Name of First Committee Member}{Dr. Name of Second Committee Member}{Dr Name of Third Committee Member}

\chapter*{Abstract}
Your abstract goes here. Blah blah blah.


\chapter*{Dedication}
I dedicate this thesis to my pets, Sparky and Coco. 


\certificateoforiginality

\chapter*{Acknowledgment}
I would like to thank my advisor, my parents, my friends and my pets. Blah blah blah ...

\tableofcontents
\listoffigures
\listoftables
% \lstlistoflistings

\resetpagenumbering


%%%%%%%%%%%%%%%%%%%%%%%%%%%%%%%%%%%%%%%%%%%%%%%%%%%%%%%%%%%


\chapter{Introduction or the Name of your First Chapter}
\label{sec:label_of_first_chapter}

\section{Name of the First Section of Your First Section}
\label{sec:Label_of_First_Section}
%
Over the past ten years blah blah blah blah.

\section{The second section of your first chapter}
\label{sec:The_second_label}
%
There you go, getting boring already.

\subsection{First subsection}
\label{subsec:First_subsection}
%
While designing blah blah blah. We designed a classification algorithm for activity inference.



%%%%%%%%%%%%%%%%%%%%%%%%%%%%%%%%%%%%%%%%%%%%%%%%%%%%%%%%%%%


\chapter{Literature Review}
\label{ch:Literature_Review}
%
\section{Mobile Phone Sensors}
In the past few years Mobile Phones have become an essential communication device. Previously, Goodwin, Velicer and Intille \cite{Goodwin2008} used cell phones to record feelings and opinions of participants in behavioral studies, while \cite{Ahas2007} tracks recreation and tourism behavior by GPS enabled cell phones. Using cell phones as part of the study is unlikely to introduce health bias in a data set according to Lajunen et al. \cite{Lajunen2007} furthermore, computer use did correlate with higher BMI results in young people, while cell phone use was only weakly linked.

Table \ref{tab:Activity_Traces} shows the number of activity traces collected for various activities, broken up by the phone placement.

\begin{table}[!tbh]
\renewcommand{\arraystretch}{1.5}
\caption{Data set of activity traces.}
\label{tab:Activity_Traces}
\centering
\begin{tabular}[width=\columnwidth]{|p{0.9in}||p{0.5in}|p{0.5in}|p{0.5in}|p{0.5in}|p{0.5in}|}
\hline
Activity $\backslash$ Placement & Pant Pocket & Hand & Hand Bag & Shirt Pocket & Sub-total \\
\hline
Walking             & 25 & 20 & 15 & 20 & \textbf{80} \\
Running             & 20 & 20 & 25 & 15 & \textbf{80} \\
Climbing Stairs     & 15 & 25 & 15 & 10 & \textbf{75} \\
Descending Stairs   & 15 & 20 & 15 & 10 & \textbf{60} \\
Driving             & 10 & 10 & 20 & 25 & \textbf{65} \\
Cycling             & 25 & 15 & 15 & 10 & \textbf{65} \\
Inactive            & 20 & 20 & 25 & 20 & \textbf{85} \\
\hline
\textbf{TOTAL}      &    &    &    &    & \textbf{510}\\
\hline
\end{tabular}
\end{table}


\begin{equation}
%\begin{aligned}
\sigma_{ij} = \frac{1}{N-1} \sum_{n=1}^{N}{\left(a_{i}[n]-\overline{a}_i\right)\left(a_j[n]-\overline{a}_j]\right)} \\
\label{eq:Covariance}
%\end{aligned}
\end{equation}

The covariance matrix $\mathbf{C}$ is defined in terms of covariance terms as,

\begin{equation}
%\begin{aligned}
\mathbf{C} = \left[\begin{array}{ccc}
              \sigma_x^2 & \sigma_{xy} & \sigma_{xz} \\
              \sigma_{yx} & \sigma_y^2 & \sigma_{yz} \\
              \sigma_{zx} & \sigma_{zy} & \sigma_z^2
              \end{array} \right] \\
\label{eq:Covariance_Matrix}
%\end{aligned}
\end{equation}

If $\mathbf{V}$ is the matrix of eigenvectors, then the matrix $\mathbf{A}$ of accelerometer signals is transformed into matrix $\mathbf{D}$ according to Equation \ref{eq:Coord_Transformation}.


\begin{equation}
%\begin{align}
\mathbf{D} = \mathbf{A} \mathbf{V}
\label{eq:Coord_Transformation}
\end{equation}
%
where
%
\begin{equation}
\mathbf{D} = \left[\begin{array}{ccc}d_1[n] & d_2[n] & d_3[n] \end{array}\right]
\mathbf{A} = \left[  \begin{array}{ccc}a_x[n] & a_y[n] & a_z[n]\end{array} \right]
%\end{align}
\end{equation}

%%%%%%%%%%%%%%%%%%%%%%%%%%%%%%%%%%%%%%%%%%%%%%%%%%%%%%%%%

\chapter{Implementation and Results}
\label{ch:Implementation_and_Results}

\section{Classifier Training \& Validation Strategy}
\label{subsec:Classifier_Training_and_Validation_Strategy}

\subsection{Classifier Training}
\label{subsec:Classifier_Training}
%
Machine learning algorithms are used to classify activities based on a feature set.


\begin{figure}[htp]
\centering
\includegraphics[width=0.9\columnwidth]{nust}%{drivingpeaksrsq}
\caption{Feature rank by information gain.}
\label{fig:Feature_Rank}
\end{figure}


\section{Performance Evaluation}
\label{sec:Performance_Evaluation}
%
For performance evaluation we used 10-fold validation.

\begin{table}[!tbh]
%\renewcommand{\arraystretch}{1.5}
\caption{Detailed Accuracy By Class (Na\"ive Bayes).}
\label{tab:Perf_Metrics_Naive_Bayes}
\centering
\begin{tabular}[width=\columnwidth]{|p{1.3in}|c|c|c|c|c|}
\hline
Class               & TP Rate   & FP Rate   & Precision & Recall    & ROC Area \\
\hline
Walking 	        	& 0.789	    & 0.031	    & 0.675	    & 0.789	    & 0.818 \\
Running	            & 1.000	    & 0.000	    & 1.000	    & 1.000	    & 0.999 \\
Climbing Stairs	    & 0.450	    & 0.054	    & 0.731	    & 0.450	    & 0.807 \\
Descending Stairs		& 0.833	    & 0.023	    & 0.814	    & 0.833	    & 0.919 \\
Driving	            & 0.933	    & 0.018	    & 0.897	    & 0.933	    & 0.934 \\
Cycling	            & 1.000	    & 0.000	    & 0.880	    & 1.000	    & 0.992 \\
Inactive            & 0.933	    & 0.000	    & 0.996	    & 0.933	    & 0.961 \\
\hline
\textbf{Weighted Average}	& \textbf{0.847}	    & \textbf{0.040}	    & \textbf{0.846}	    & \textbf{0.847}	    & \textbf{0.891} \\
\hline
\end{tabular}
\end{table}

\begin{table}[!tbh]
%\renewcommand{\arraystretch}{1.5}
\caption{Confusion matrix (Naive Bayes).}
\label{tab:Confusion_Matrix_Naive Bayes}
\centering
\begin{tabular}[width=\columnwidth]{|ccccccc|p{1.5in}|}
\hline
a & b & c & d & e & f & g & $\leftarrow$ Classified As \\
  &   &   &   &   &   &   & Actual Activity $\downarrow$ \\
\hline
\textbf{59} & 0 & 6 & 5 & 1 & 0 & 0 & a $\leftarrow$ Walking \\
0 & \textbf{50} & 0 & 1 & 0 & 0 & 0 & b $\leftarrow$ Running \\
5 & 0 & \textbf{34} & 4 & 0 & 0 & 0 & c $\leftarrow$ Climbing Stairs \\
4 & 0 & 3 & \textbf{29} & 5 & 0 & 0 & d $\leftarrow$ Descending Stairs \\
0 & 0 & 0 & 0 & \textbf{32} & 0 & 4 & e $\leftarrow$ Driving \\
2 & 0 & 0 & 1 & 0 & \textbf{20} & 0 & f $\leftarrow$ Cycling \\
0 & 0 & 0 & 0 & 5 & 0 & \textbf{30} & g $\leftarrow$ Inactive \\
\hline
\end{tabular}
\end{table}

%%%%%%%%%%%%%%%%%%%%%%%%%%%%%%%%%%%%%%%%%%%%%%%%%%%%%%%%%%%

\chapter{Conclusions}
\label{ch:Conclusions}

\section{This is the End}
\label{sec:This_is_the_end}
%
In this research we reported the design and implementation of.

%%%%%%%%%%%%%%%%%%%%%%%%%%%%%%%%%%%%%%%%%%%%%%%%%%%%%%%%%%%

\appendix
\chapter{Feature Extraction Walking}
%
Features plotted in Matlab.


\bibliographystyle{plain}
\bibliography{thesis}


\backmatter


\end{document}
