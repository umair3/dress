% Based on template for ICASSP-2008 paper; to be used with:
%          spconf.sty  - ICASSP/ICIP LaTeX style file, and
%          IEEEbib.bst - IEEE bibliography style file.
% --------------------------------------------------------------------------
\documentclass[a4paper]{article} 
\usepackage{spconf,amsmath,epsfig}

% Example definitions.
% --------------------
\def\x{{\mathbf x}}
\def\L{{\cal L}}

% Title.
% ------
\title{AUTHOR PAPER PROPOSAL TEMPLATE AND GUIDELINES \\ 
       FOR PAPERS IN THE KALEIDOSCOPE PROCEEDINGS}
%
% Single address - DO NOT USE FOR PAPER PROPOSAL!!!.
% ---------------------------------------------------
\name{\thanks{Thanks to XYZ agency for funding - if any.}}
\address{} % Leave this empty for the paper proposal phase!
%
%


\begin{document}

% Change page numbering style *only* for the paper proposal phase!
\pagestyle{plain}
\pagenumbering{arabic}

%\ninept
%
\maketitle
%
\begin{abstract} \em
The abstract should appear in \underline{italics} at the top of the
left-hand column of text, about 12 mm (0.5 inch) below the title area
and no more than 80 mm (3.125 inches) in length.  Leave a 12 mm (0.5
inch) space between the end of the abstract and the beginning of the
main text.  The abstract should normally contain 100 to 150 words, and
in no case it shall exceed 200 words. The abstract must be identical
to the abstract text submitted electronically in EDAS, since it is the
latter that will be used for generating the table of abstracts for the
conference, not the abstract and title in the paper itself.  All
manuscripts must be in English, printed in black ink.
\end{abstract}
%
\begin{keywords}
One, two, three, four, five
\end{keywords}
%
\section{Introduction}
\label{sec:intro}

These guidelines include complete descriptions of the number of pages,
layout, fonts, spacing, and related information for producing your
proceedings manuscripts. Please follow them and if you have any
questions, direct them to the Kaleidoscope secretariat at
{\em kaleidoscope@itu.int}. Papers not adhering to them will be returned to
you and if proper formatting is not achieved by the deadline this will
lead to the paper being rejected for inclusion in the proceedings and
presentation at the conference.

\section{Formatting your paper}
\label{sec:format}

Your paper {\bf cannot} have more than eight pages (there included all
text, all figures, tables, abstract and references). All printed
material, including text, illustrations, and charts must be kept
within a print area of 175 mm (6.9 inches) wide by 244 mm (9.6 inches)
high. Do not write or print anything outside the print area. The top
margin must be 25 mm (1 inch), except for the title page, and the left
margin must be 17.5 mm (0.69 inch).  All {\it text} must be in a
two-column format. Columns are to be 85 mm (3.35 inches) wide, with a
5 mm (0.20 inch) space between them. Text must be fully
justified. Units should be expressed as much as possible in
international units, and a dot (``.'') should be used to express
decimal points (not ``,'').

\section{PAGE TITLE SECTION}
\label{sec:pagestyle}

The paper title (on the first page) should begin 35 mm (1.38 inches)
from the top edge of the page, centered, completely capitalized, and
in Times-Roman 12-point, boldface type.  For the PAPER PROPOSAL phase,
the authors' name(s) and affiliation(s) SHOULD NOT appear below the
title, nor anywhere else in the paper.\\
{\small NOTE: This instruction will {\bf change} for the camera-ready
version of accepted papers. In that case, the authors' name(s) and
affiliation(s) are to appear below the title in capital and lower case
letters.  Papers with multiple authors and affiliations may require
two or more lines for this information.}

\section{TYPE-STYLE AND FONTS}
\label{sec:typestyle}

To achieve the best rendering both in the proceedings and from the
CD-ROM, we strongly encourage you to use Times-Roman font.  In
addition, this will give the proceedings a more uniform look.  Use a
font that is {\bf no} smaller than nine point type throughout the
paper, including figure captions.

In nine point type font, capital letters are 2 mm high.  If you use the
smallest point size, there should be no more than 3.2 lines/cm (8 lines/inch)
vertically.  This is a minimum spacing; 2.75 lines/cm (7 lines/inch) will make
the paper much more readable.  Larger type sizes require correspondingly larger
vertical spacing.  Please do not double-space your paper.  True-Type 1 fonts
are preferred.

\section{MAJOR HEADINGS}
\label{sec:majhead}

Major headings (for example, ``1. Introduction'') should appear in all
capital letters, bold face if possible, centered in the column, with
one blank line before, and one blank line after. Use a period (``.'')
after the heading number, not a colon.

\subsection{Subheadings}
\label{ssec:subhead}

Subheadings should appear in lower case (initial word capitalized) in
boldface.  They should start at the left margin on a separate line.
 
\subsubsection{Sub-subheadings}
\label{sssec:subsubhead}

Sub-subheadings, as in this paragraph, are discouraged. However, if you
must use them, they should appear in lower case (initial word
capitalized) and start at the left margin on a separate line, with paragraph
text beginning on the following line.  They should be in italics.

\section{PRINTING YOUR PAPER}
\label{sec:print}

Print your properly formatted text on high-quality, A4 size (210 mm
wide by 297 mm long, or 8.27 inches by 11.7 inches). If the last page
of your paper is only partially filled, arrange the columns so that
they are evenly balanced if possible, rather than having one long
column.

In LaTeX, to start a new column (but not a new page) and help balance the
last-page column lengths, you can use the command ``$\backslash$pagebreak'' as
demonstrated on this page (see the LaTeX source below).

The paper proposal must be submitted either as a Post\-Script (PS)
or unprotected Adobe's Portable Document Format (PDF) file. All fonts
{\bf must} be embedded and the file should contain {\bf no}
bookmarks. These are {\em strict} publisher's requirements. Look for
information regarding embedding fonts and bookmark generation in the
EDAS help page at {\em
http://edas.info/listFAQ.php?c=19927}. Additional hints can be found in
the IEEE PDF Express website at:
http://www.pdf-express.org/pdfcheck.asp. \\
{\small NOTE: The IEEE PDF Express site will be available for production of the camera-ready version of accepted papers.}


\section{ILLUSTRATIONS, GRAPHS, AND PHOTOGRAPHS}
\label{sec:illust}

All halftone illustrations must be clear black and white prints.  Do
not use any colors in illustrations, since the proceedings will be
printed in black and white. It is your responsibility to ensure that
illustrations (there included pictures, diagrams, etc) are properly
rendered in a black-and-white laser printer. 

Illustrations must appear within the designated margins.  They may
span the two columns.  If possible, position illustrations at the top
of columns, rather than in the middle or at the bottom.  Caption and
number every illustration. Figure captions must be placed on the {\bf
bottom}, not top, of the figure. Table captions must be located on the
{\bf top} of the table.

Since there are many ways, often incompatible, of including images (e.g., with
experimental results) in a LaTeX document, below is an example of how to do
this \cite{Lamp86}.

% Below is an example of how to insert images. Delete the ``\vspace'' line,
% uncomment the preceding line ``\centerline...'' and replace ``imageX.ps''
% with a suitable PostScript file name.
% -------------------------------------------------------------------------
\begin{figure}[htb]

\begin{minipage}[b]{1.0\linewidth}
  \centering
% \centerline{\epsfig{figure=image1.ps,width=8.5cm}}
  %\vspace{2.0cm}
  \begin{picture}(60,60)\put(2,2){\framebox(50,50)[c]{Square}}\end{picture}
  \centerline{(a) Result 1}\medskip
\end{minipage}
%
\begin{minipage}[b]{.48\linewidth}
  \centering
% \centerline{\epsfig{figure=image3.ps,width=4.0cm}}
  %\vspace{1.5cm}
  \begin{picture}(60,50)\put(30,30){\circle{50}}\end{picture}
  \centerline{(b) Result 2}\medskip
\end{minipage}
\hfill
\begin{minipage}[b]{0.48\linewidth}
  \centering
% \centerline{\epsfig{figure=image4.ps,width=4.0cm}}
  %\vspace{1.5cm}
  \begin{picture}(60,50)\put(30,35){\oval(30,20)}\end{picture}
  \centerline{(c) Result 3}\medskip
\end{minipage}
%
\caption{Example of placing a figure with experimental results.}
\label{fig:res}
%
\end{figure}

% To start a new column (but not a new page) and help balance the last-page
% column length use \vfill\pagebreak.
% -------------------------------------------------------------------------
\vfill
\pagebreak


\section{PAGE NUMBERING}
\label{sec:page}

Please number all pages in your paper proposal. Note that this
instruction will {\bf change} for the camera-ready version.

\section{FOOTNOTES}
\label{sec:foot}

Use footnotes sparingly (or not at all!) and place them at the bottom of the
column on the page on which they are referenced. Use Times 9-point type,
single-spaced. To help your readers, avoid using footnotes altogether and
include necessary peripheral observations in the text (within parentheses, if
you prefer, as in this sentence).

\section{USING REFERENCES}
\label{sec:ref}

List and number all bibliographical references at the end of the
paper.  The references can be numbered in alphabetic order or in order
of appearance in the document.  When referring to them in the text,
type the corresponding reference number in square brackets as shown at
the end of this sentence \cite{C2}.

% References should be produced using the bibtex program from suitable
% BiBTeX files (here: strings, refs, manuals). The IEEEbib.bst bibliography
% style file from IEEE produces unsorted bibliography list.
% -------------------------------------------------------------------------
\bibliographystyle{IEEEbib}
\bibliography{strings,refs}

\end{document}
